\documentclass[a4paper]{article}
\usepackage[left=20mm, right=20mm]{geometry}

\usepackage{newtxtext, newtxmath}
\usepackage{bm}
\usepackage{amsmath}
\usepackage{ifthen}
\usepackage{xcolor}

\title{Fundamentals on the package}
\author{Yuu Miino}
\date{\today}

\newcommand{\R}{\mathbb{R}}
\newcommand{\deriv}[3][1]{%
    \ifthenelse{\equal{#1}{1}}{\frac{d #2}{d #3}}{\frac{d^#1 #2}{{d #3}^#1}}
}
\newcommand{\pderiv}[3][1]{%
    \ifthenelse{\equal{#1}{1}}{\frac{\partial #2}{\partial #3}}{\frac{\partial^#1 #2}{{\partial #3}^#1}}
}
\newcommand{\jac}{J}
\newcommand{\hes}{H}
\newcommand{\parens}[1]{\left(#1\right)}
\newcommand{\tr}{\mathop{\mathrm{tr}}\nolimits}
\newcommand{\braces}[1]{\left\{#1\right\}}
\newcommand{\set}[1]{\braces{#1}}
\newcommand{\parm}{\partial\!M}
\newcommand{\brackets}[1]{\left[#1\right]}
\newcommand{\traj}{\bm \varphi}
\newcommand{\red}[1]{\textcolor{red}{#1}}

\begin{document}
\maketitle
\section{Derivatives of the composite map}
Let $V$, $W$, and $X$ be vector spaces over $\R$,
and $f$ and $g$ be diffeomorphisms defined by
\begin{align}
    \begin{aligned}
        f&: V \to W, &
        g&: W \to X.
    \end{aligned}
\end{align}
Then, the Jacobian matrix of $f$ and $g$ are elements of tensor products
\begin{align}
    \begin{aligned}
        \jac_f &\in W \otimes V^*, &
        \jac_g &\in X \otimes W^*,
    \end{aligned}
\end{align}
where $V^*$ and $W^*$ are the dual spaces of $V$ and $W$, respectively.
$\jac_f$ is also a map $V \to W$; $\jac_g$ is $W \to X$.
If $\jac_f$ and $\jac_g$ are differentiable in $V$ and $W$,
the Hessian tensors $\hes_f$ and $\hes_g$ are available as elements of another tensor products
\begin{align}
    \begin{aligned}
        \hes_f &\in W \otimes V^* \otimes V^*, &
        \hes_g &\in X \otimes W^* \otimes W^*.
    \end{aligned}
\end{align}
On the other hand, a composite map $g\circ f$ is described as
\begin{align}
    g \circ f: V \to X.
\end{align}
From the chain-rule, the Jacobian matrix of $g\circ f$ is
\begin{align}
    \jac_{g\circ f} = \jac_g \jac_f \in X \otimes V^*.
    \label{eq:jacgf}
\end{align}
In the context of the tensor, Eq. (\ref{eq:jacgf}) is equivalent to
the following contraction of the tensor product.
\begin{align}
    \jac_{g\circ f} = \tr_{23} \parens{
        \jac_g \otimes \jac_f
    },
\end{align}
where
\begin{align}
    \begin{aligned}
        \jac_g \otimes \jac_f &\in \parens{X \otimes W^*} \otimes \parens{W \otimes V^*}, \\
        \tr_{23} & : \parens{X \otimes W^*} \otimes \parens{W \otimes V^*} \to
        X \otimes V^*.
    \end{aligned}
\end{align}
We write the $(i, j)$ contraction of a tensor by $\tr_{ij}$,
which is a generalization of the trace.
The Hessian tensor of $g\circ f$ is an element of
$X \otimes V^* \otimes V^*$.
From the chain-rule, we have
\begin{align}
    \hes_{g\circ f} =
    \parens{\hes_g \jac_f} \jac_f +
    \jac_g \hes_f,
\end{align}
where
\begin{align}
    \begin{aligned}
        \hes_g \jac_f & = \tr_{24}\parens{\hes_g \otimes \jac_f}
        \quad\parens{= \tr_{34}\parens{\hes_g \otimes \jac_f}},\\
        \hes_g \otimes \jac_f &\in (X \otimes W^* \otimes W^*) \otimes (W \otimes V^*),\\
        \tr_{24}\parens{= \tr_{34}} &: (X \otimes W^* \otimes W^*) \otimes (W \otimes V^*) \to
        X \otimes W^* \otimes V^*,
    \end{aligned}
    \\[15pt]
    \begin{aligned}
        \parens{\hes_g \jac_f} \jac_f &= \tr_{24}\parens{\parens{\hes_g \jac_f} \otimes \jac_f},\\
        \tr_{24} &: \parens{X \otimes W^* \otimes V^*} \otimes \parens{W \otimes V^*}
        \to X \otimes V^* \otimes V^*,
    \end{aligned}
    \\[15pt]
    \begin{aligned}
        \jac_g \hes_f &= \tr_{23}\parens{\jac_g \otimes \hes_f},\\
        \jac_g \otimes \hes_f &\in (X \otimes W^*) \otimes (W \otimes V^* \otimes V^*),\\
        \tr_{23} &: (X \otimes W^*) \otimes (W \otimes V^* \otimes V^*) \to
        X \otimes V^* \otimes V^*,
    \end{aligned}
\end{align}


Let $T$ be a composite map of $T_i$
\begin{align}
    \begin{aligned}
        T &= T_{m-1} \circ T_{m-2} \circ \cdots \circ T_1 \circ T_0,
        &
        T: & M_0 \to M_{m-1}
    \end{aligned}
    \end{align}
where $T_i$ is a $C^{\infty}$ diffeomorphism $T_i: M_i \to M_{i+1}$ and $M_i \subset \R$.
Given $\bm x_i \in M_i$, the Jacobian matrix of $T$ is
\begin{align}
    \jac = \pderiv{T}{\bm x_0} = \prod_{k=0}^{m-1} \pderiv{T_{m-1-k}}{\bm x_{m-1-k}}.
\end{align}
Let us denote the product in the right-hand of equation as $\jac_{m-1}$, and we have
\begin{align}
    \jac_{m-1} = \pderiv{T_{m-1}}{\bm x_{m-1}}\, \jac_{m-2},
    \label{eq:jac}
\end{align}
for $m \ge 2$.
If $\jac_{k}$ is
Hessian tensor of $T$ is available by differentiating Eq. (\ref{eq:jac}),
\begin{align}
    \hes = \pderiv{\jac}{\bm x_0}
    =
    \parens{
        \pderiv[2]{T_{m-1}}{\bm x_{m-1}}\,
        \jac_{m-2}
    }
    \jac_{m-2}
    +
    \pderiv{T_{m-1}}{\bm x_{m-1}}
    \pderiv{\jac_{m-2}}{\bm x_0}.
\end{align}
Rewriting the derivative of $\jac_{m-1}$ by $\hes_{m-1}$, we get
\begin{align}
    \hes_{m-1} =
    \parens{
        \pderiv[2]{T_{m-1}}{\bm x_{m-1}}\,
        \jac_{m-2}
    }
    \jac_{m-2}
    +
    \pderiv{T_{m-1}}{\bm x_{m-1}}
    \hes_{m-2}
\end{align}
Notice that $\parens{\jac_{m-2}\, \jac_{m-2}}$ is sometimes not $\jac_{m-2}^2$
because the dimensions of $M_{i}$ and $M_{i+1}^*$
are not necessarily equal.
\clearpage
\section{Derivative of a map for the continuous-time systems}
Consider a $C^{\infty}$ autonomous dynamical system
\begin{align}
    \begin{aligned}
        \deriv{\bm x}{t} &= \bm f(\bm x), &
        \bm x &\in M \subset \R^n, &
        t &\in \R,
    \end{aligned}
    \label{eq:sys}
\end{align}
where $M$ is a state space, and $\bm f$ is a function such that $M \to \R^n$.
We write the trajectory of the system (\ref{eq:sys}) by $\traj: \R \times M \to M$,
where $\traj(0, \bm x_0) = \bm x_0$ is the initial state
and $\traj(t, \bm x_0)$ is the state at $t$.
Let $\parm$ be an $n-1$ dimensional manifold defined by
a conditional function $q: M \to \R$
\begin{align}
    \parm = \set{
        \bm x \in M ~|~
        q(\bm x) = 0
    }
\end{align}
Suppose that $T_0$ is a local map from $\bm x_0 \in M$ to a point in $\parm$
such that $M \to \parm$.
Then, the Jacobian matrix of $T_0$ is described by
\begin{align}
    \jac_{T_0} = \pderiv{T_0}{\bm x_0} =
    \left.
    \brackets{
        I -
        \frac{1}{\deriv{q}{\bm x} \bm f(\bm x)}
        \bm f(\bm x) \deriv{q}{\bm x}
    }
    \right|_{\bm x = \bm x_1}
    \pderiv{\traj}{\bm x_0}(\tau)
    =
    B(\bm x_1) \pderiv{\traj}{\bm x_0}(\tau),
\end{align}
where $\tau$ is the spent time during the trajectory $\traj$
moves from $\bm x_0$ to the boundary $\parm$, which only depends on $\bm x_0$,
and $\bm x_1 = \traj(\tau, \bm x_0)$.
The Hessian tensor of $T_0$ is
\begin{align}
    \hes_{T_0}
    =
    \parens{
        \pderiv{B}{\bm x} (\bm x_1) \pderiv{\traj}{\bm x_0}(\tau)
    }
    \pderiv{\traj}{\bm x_0}(\tau)
    +
    B(\bm x_1) \pderiv[2]{\traj}{\bm x_0}(\tau),
\end{align}
where
\begin{align}
    \pderiv{B}{\bm x}
    =
    - \frac{1}{
        \parens{\deriv{q}{x} \bm f(\bm x)}^2
    }
    \braces{
        \parens{
            \deriv{\bm f}{\bm x} \otimes \deriv{q}{\bm x}
            +
            \bm f(\bm x) \otimes \deriv[2]{q}{\bm x}
        }
        \deriv{q}{\bm x} \bm f(\bm x)
        -
        \parens{\bm f(\bm x) \deriv{q}{\bm x}} \otimes
        \parens{
            \deriv[2]{q}{\bm x} \bm f(\bm x)
            +
            \deriv{q}{\bm x} \deriv{\bm f}{\bm x}
        }
    },
\end{align}
since
\begin{align}
    \begin{aligned}
        B &\in X \otimes X^* ,&
        \pderiv{B}{\bm x} &\in X \otimes X^* \otimes X^*,\\
        \deriv{q}{\bm x}&\in X^*, &
        \deriv[2]{q}{\bm x} & \in X^* \otimes X^*,\\
        \bm f(\bm x) &\in X, &
        \deriv{\bm f}{\bm x} &\in X \otimes X^*.
    \end{aligned}
\end{align}

\red{Check if order ``F'' or ``C''!}

On the other hand,
\begin{align}
    \deriv{}{t}
    \parens{\pderiv{\traj}{\bm x_0}}
    &= \deriv{\bm f}{\bm x} \pderiv{\traj}{\bm x_0}\\
    \deriv{}{t}
    \parens{\pderiv[2]{\traj}{\bm x_0}}
    &=
    \parens{\deriv[2]{\bm f}{\bm x} \pderiv{\traj}{\bm x_0}}
    \pderiv{\traj}{\bm x_0}
    +
    \deriv{\bm f}{\bm x} \pderiv[2]{\traj}{\bm x_0},
\end{align}
where
\begin{align}
    \begin{aligned}
        \deriv[2]{\bm f}{\bm x} \pderiv{\traj}{\bm x_0}
        &= \tr_{34} \parens{
            \deriv[2]{\bm f}{\bm x} \otimes \pderiv{\traj}{\bm x_0}
        },
        &
        \deriv{\bm f}{\bm x} \pderiv[2]{\traj}{\bm x_0}
        &=
        \tr_{23} \parens{
            \deriv{\bm f}{\bm x} \otimes \pderiv[2]{\traj}{\bm x_0}
        }.
    \end{aligned}
\end{align}

\clearpage
\begin{align}
    &\begin{aligned}
        \pderiv{T_0}{\bm x_0}
        &= \pderiv{\traj}{\bm x_0}
        + \bm f(\bm x_1) \pderiv{\tau}{\bm x_0}
        \\
        \pderiv{\tau}{\bm x_0}
        &=
        - \frac{1}{\deriv{q}{\bm x} \bm f(\bm x_1)} \deriv{q}{\bm x} \pderiv{\traj}{\bm x_0}
        \\
        \pderiv{T_0}{\bm x_0}
        &=
        \brackets{
            I - \frac{1}{\deriv{q}{\bm x} \bm f(\bm x_1)} \bm f(\bm x_1) \deriv{q}{\bm x}
        } \pderiv{\traj}{\bm x_0}
        = B(\bm x_1) \pderiv{\traj}{\bm x_0}
    \end{aligned}
    \\[20pt]
    &\begin{aligned}
        \pderiv[2]{T_0}{\bm x_0} &=
        \pderiv[2]{\traj}{\bm x_0}
        +
        \parens{
            \deriv{\bm f}{\bm x}
            \pderiv{\traj}{\bm x_0}
        } \otimes
        \pderiv{\tau}{\bm x_0}
        +
        \bm f(\bm x_1) \otimes
        \pderiv[2]{\tau}{\bm x_0}
        \\
        \pderiv[2]{\tau}{\bm x_0} &=
        - \frac{1}{
            \parens{\deriv{q}{\bm x}\bm f(\bm x_1)}^2
        }
        \brackets{
            \pderiv{}{\bm x_0}\parens{\deriv{q}{\bm x}\pderiv{\traj}{\bm x_0}}
            \deriv{q}{\bm x} \bm f(\bm x_1)
            -
            \parens{\deriv{q}{\bm x}\pderiv{\traj}{\bm x_0}} \otimes
            \pderiv{}{\bm x_0}\parens{
                \deriv{q}{\bm x} \bm f(\bm x_1)
            }
        }
        \\
        &=
        - \frac{1}{
            \parens{\deriv{q}{\bm x}\bm f(\bm x_1)}^2
        }
        \braces{
            \brackets{
            \parens{\deriv[2]{q}{\bm x}\pderiv{\traj}{\bm x_0}}
            \pderiv{\traj}{\bm x_0}
            +
            \deriv{q}{\bm x}\pderiv[2]{\traj}{\bm x_0}
            }
            \deriv{q}{\bm x} \bm f(\bm x_1)
            -
            \parens{\deriv{q}{\bm x}\pderiv{\traj}{\bm x_0}}\otimes
            \brackets{
                \parens{
                    \deriv[2]{q}{\bm x}\pderiv{\traj}{\bm x_0}
                }
                \bm f(\bm x_1)
                +
                \deriv{q}{\bm x}
                \parens{\deriv{\bm f}{\bm x} \pderiv{\traj}{\bm x_0}}
            }
        }
        \\
        &=
        - \frac{1}{
            {\deriv{q}{\bm x}\bm f(\bm x_1)}
        }
        \brackets{
            \parens{\deriv[2]{q}{\bm x}\pderiv{\traj}{\bm x_0}}
            \pderiv{\traj}{\bm x_0}
            +
            \deriv{q}{\bm x}\pderiv[2]{\traj}{\bm x_0}
            }
        + \frac{1}{
            \parens{\deriv{q}{\bm x}\bm f(\bm x_1)}^2
        }
        \parens{\deriv{q}{\bm x}\pderiv{\traj}{\bm x_0}}\otimes
        \brackets{
            \parens{
                \deriv[2]{q}{\bm x}\pderiv{\traj}{\bm x_0}
            }
            \bm f(\bm x_1)
            +
            \deriv{q}{\bm x}
            \parens{\deriv{\bm f}{\bm x} \pderiv{\traj}{\bm x_0}}
        }
        \\
        \pderiv[2]{T_0}{\bm x_0}
        &= B(\bm x_1) \pderiv[2]{\traj}{\bm x_0}
        - \frac{1}{\deriv{q}{\bm x} \bm f(\bm x_1)}
        \brackets{
            \parens{
                \deriv{\bm f}{\bm x}
                \pderiv{\traj}{\bm x_0}
            }
            \otimes \deriv{q}{\bm x}
            +
            \bm f(\bm x_1) \otimes \parens{\deriv[2]{q}{\bm x}\pderiv{\traj}{\bm x_0}}
        } \pderiv{\traj}{\bm x_0}
        \\
        &\qquad
        + \frac{1}{
            \parens{\deriv{q}{\bm x}\bm f(\bm x_1)}^2
        } \bm f(\bm x_1) \otimes
        \parens{\deriv{q}{\bm x}\pderiv{\traj}{\bm x_0}}\otimes
        \brackets{
            \parens{
                \deriv[2]{q}{\bm x}\pderiv{\traj}{\bm x_0}
            }
            \bm f(\bm x_1)
            +
            \deriv{q}{\bm x}
            \parens{\deriv{\bm f}{\bm x} \pderiv{\traj}{\bm x_0}}
        }
        \\
        &= B(\bm x_1) \pderiv[2]{\traj}{\bm x_0}
        - \frac{1}{\deriv{q}{\bm x} \bm f(\bm x_1)}
        \brackets{\parens{
            \deriv{\bm f}{\bm x}
            \otimes \deriv{q}{\bm x}
            +
            \bm f(\bm x_1) \otimes \deriv[2]{q}{\bm x}
        }\pderiv{\traj}{\bm x_0}}  \pderiv{\traj}{\bm x_0}
        \\
        &\qquad
        + \frac{1}{
            \parens{\deriv{q}{\bm x}\bm f(\bm x_1)}^2
        }
        \parens{\bm f(\bm x_1) \deriv{q}{\bm x}}\otimes
        \brackets{\parens{
            \deriv[2]{q}{\bm x} \bm f(\bm x_1)
            +
            \deriv{q}{\bm x} \deriv{\bm f}{\bm x}
        }\pderiv{\traj}{\bm x_0}} \pderiv{\traj}{\bm x_0}
        \\
        &= B(\bm x_1) \pderiv[2]{\traj}{\bm x_0}
        \\
        &\quad
        - \frac{1}{\parens{\deriv{q}{\bm x} \bm f(\bm x_1)}^2}
        \braces{\brackets{\parens{
            \deriv{\bm f}{\bm x}
            \otimes \deriv{q}{\bm x}
            +
            \bm f(\bm x_1) \otimes \deriv[2]{q}{\bm x}
        }
        \deriv{q}{\bm x} \bm f(\bm x_1)
        -
        \parens{\bm f(\bm x_1) \deriv{q}{\bm x}}\otimes
        \parens{
            \deriv[2]{q}{\bm x} \bm f(\bm x_1)
            +
            \deriv{q}{\bm x} \deriv{\bm f}{\bm x}
        }}\pderiv{\traj}{\bm x_0}} \pderiv{\traj}{\bm x_0}
        \\
        &=
        B(\bm x_1) \pderiv[2]{\traj}{\bm x_0} +
        \parens{\deriv{B}{\bm x} \pderiv{\traj}{\bm x_0}} \pderiv{\traj}{\bm x_0}
    \end{aligned}
\end{align}

\end{document}
