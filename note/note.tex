\documentclass[a4paper]{article}
\usepackage[left=20mm, right=20mm, top=25mm, bottom=25mm]{geometry}

\usepackage{newtxtext, newtxmath}
\usepackage{bm}
\usepackage{amsmath}
\usepackage{ifthen}
\usepackage{xcolor}
\usepackage{listings}
\definecolor{codegreen}{rgb}{0,0.6,0}
\definecolor{codegray}{rgb}{0.5,0.5,0.5}
\definecolor{codepurple}{rgb}{0.58,0,0.82}
\definecolor{backcolor}{rgb}{0.95,0.95,0.92}
\lstdefinestyle{code}{
    backgroundcolor=\color{backcolor},
    commentstyle=\color{codegreen},
    keywordstyle=\color{magenta},
    stringstyle=\color{codepurple},
    basicstyle=\ttfamily\footnotesize,
    breakatwhitespace=false,
    breaklines=true,
    captionpos=b,
    keepspaces=true,
    showspaces=false,
    showstringspaces=false,
    showtabs=false,
    tabsize=2,
    frame=trBL,
    xleftmargin=.1\hsize, xrightmargin=.1\hsize,
    abovecaptionskip=12pt
}
\lstset{style=code}

\title{Fundamentals on the package}
\author{Yuu Miino}
\date{\today}

\newcommand{\R}{\mathbb{R}}
\newcommand{\deriv}[3][1]{%
    \ifthenelse{\equal{#1}{1}}{\frac{d #2}{d #3}}{\frac{d^#1 #2}{{d #3}^#1}}
}
\newcommand{\pderiv}[3][1]{%
    \ifthenelse{\equal{#1}{1}}{\frac{\partial #2}{\partial #3}}{\frac{\partial^#1 #2}{{\partial #3}^#1}}
}
\newcommand{\jac}{J}
\newcommand{\hes}{H}
\newcommand{\parens}[1]{\left(#1\right)}
\newcommand{\tr}{\mathop{\mathrm{tr}}\nolimits}
\newcommand{\braces}[1]{\left\{#1\right\}}
\newcommand{\set}[1]{\braces{#1}}
\newcommand{\parm}{\partial\!M}
\newcommand{\brackets}[1]{\left[#1\right]}
\newcommand{\traj}{\bm \varphi}
\newcommand{\red}[1]{\textcolor{red}{#1}}

\begin{document}
\maketitle
\section{Derivatives of a composite map}
\subsection{Singly composite map}
Let $V$, $W$, and $X$ be vector spaces over $\R$,
and $f$ and $g$ be diffeomorphisms defined by
\begin{align}
    \begin{aligned}
        f&: V \to W, &
        g&: W \to X.
    \end{aligned}
\end{align}
Then, the Jacobian matrix of $f$ and $g$ are elements of tensor products
\begin{align}
    \begin{aligned}
        \jac_f &\in W \otimes V^*, &
        \jac_g &\in X \otimes W^*,
    \end{aligned}
\end{align}
where $V^*$ and $W^*$ are the dual spaces of $V$ and $W$, respectively.
$\jac_f$ is also a map $V \to W$; $\jac_g$ is $W \to X$.
If $\jac_f$ and $\jac_g$ are differentiable in $V$ and $W$,
the Hessian tensors $\hes_f$ and $\hes_g$ are available as elements of another tensor products
\begin{align}
    \begin{aligned}
        \hes_f &\in W \otimes V^* \otimes V^*, &
        \hes_g &\in X \otimes W^* \otimes W^*.
    \end{aligned}
\end{align}
On the other hand, a composite map $g\circ f$ is described as
\begin{align}
    g \circ f: V \to X.
\end{align}
From the chain-rule, the Jacobian matrix of $g\circ f$ is
\begin{align}
    \begin{aligned}
        \jac_{g\circ f} &= \jac_g \jac_f  &\in X \otimes V^*.
    \end{aligned}
    \label{eq:jacgf}
\end{align}
In the context of the tensor, Eq. (\ref{eq:jacgf}) is equivalent to
the following contraction of the tensor product.
\begin{align}
    \jac_{g\circ f} = \tr_{23} \parens{
        \jac_g \otimes \jac_f
    },
\end{align}
where
\begin{align}
    \begin{aligned}
        \jac_g \otimes \jac_f &\in \parens{X \otimes W^*} \otimes \parens{W \otimes V^*}, \\
        \tr_{23} & : \parens{X \otimes W^*} \otimes \parens{W \otimes V^*} \to
        X \otimes V^*.
    \end{aligned}
\end{align}
We write the $(i, j)$ contraction of a tensor by $\tr_{ij}$,
which is also known as a generalization of the trace.
The Hessian tensor of $g\circ f$ is an element of
$X \otimes V^* \otimes V^*$.
From the chain-rule, we have
\begin{align}
    \hes_{g\circ f} =
    \parens{\hes_g \jac_f} \jac_f +
    \jac_g \hes_f,
\end{align}
where
\begin{align}
    \begin{aligned}
        \hes_g \jac_f & = \tr_{24}\parens{\hes_g \otimes \jac_f}
        \quad\parens{= \tr_{34}\parens{\hes_g \otimes \jac_f}},\\
        \hes_g \otimes \jac_f &\in (X \otimes W^* \otimes W^*) \otimes (W \otimes V^*),\\
        \tr_{24}\parens{= \tr_{34}} &: (X \otimes W^* \otimes W^*) \otimes (W \otimes V^*) \to
        X \otimes W^* \otimes V^*,
    \end{aligned}
    \\[15pt]
    \begin{aligned}
        \parens{\hes_g \jac_f} \jac_f &= \tr_{24}\parens{\parens{\hes_g \jac_f} \otimes \jac_f},\\
        \tr_{24} &: \parens{X \otimes W^* \otimes V^*} \otimes \parens{W \otimes V^*}
        \to X \otimes V^* \otimes V^*,
    \end{aligned}\label{eq:hjj}
    \\[15pt]
    \begin{aligned}
        \jac_g \hes_f &= \tr_{23}\parens{\jac_g \otimes \hes_f},\\
        \jac_g \otimes \hes_f &\in (X \otimes W^*) \otimes (W \otimes V^* \otimes V^*),\\
        \tr_{23} &: (X \otimes W^*) \otimes (W \otimes V^* \otimes V^*) \to
        X \otimes V^* \otimes V^*.
    \end{aligned}\label{eq:jh}
\end{align}

In the Python implementation with NumPy, an ``@'' operator (equivalently ``numpy.matmul'') works well
to shorten the calculation code of some contractions.
Listing \ref{lst:atop} shows the example usage.
The option \texttt{axes = 0} makes the function return the tensor product of specified tensors.
\begin{center}
    \lstinputlisting[%
        firstline=1, lastline=18,
        language=python,
        caption={Example usage of ``@'' operator.},
        label={lst:atop}%
    ]{contraction.py}
\end{center}
Note that the ``@'' operator just calculates the matrix multiplication of two matrices,
which compose of the last two indices of the target tensors, for each index of the other axes.
Hence, we cannot directly use ``@'' operator with Eq. (\ref{eq:hjj}) and Eq. (\ref{eq:jh}).
In the cases, we can use ``@'' operator after swapping the axes of the tensors,
as shown in Lsts. \ref{lst:atop_swap1} and \ref{lst:atop_swap2}.
\begin{center}
    \lstinputlisting[%
    firstline=20, lastline=26,
    language=python,
    caption={Example usage of ``@'' operator after axes swapping (for Eq. (\ref{eq:hjj})).},
    label={lst:atop_swap1}%
    ]{contraction.py}
\end{center}
As commented in Lst. \ref{lst:atop_swap2},
the ``@'' operation between a matrix \texttt{Jg} and a tensor \texttt{Hf.transpose(1, 0, 2)} raise unexpected result
in the mathematical sence.
The operator yields $V^* \otimes X \otimes V^*$ from $(X\otimes W^*)$ and $(V^* \otimes W \otimes V^*)$
because it just calculates the matrix product $X \otimes V^*$ from
$X\otimes W^*$ and $W\otimes V^*$ for each element in $V^*$.
We can revert the order by \texttt{transpose} method again.
\begin{center}
    \lstinputlisting[%
    firstline=28, lastline=39,
    language=python,
    caption={Example usage of ``@'' operator after axes swapping (for Eq. (\ref{eq:jh})).},
    label={lst:atop_swap2}%
    ]{contraction.py}
\end{center}
One can use the Einstein notation instead of a complex combination of traces and transposes,
as shown in Lst. \ref{lst:einsum}.
\begin{center}
    \lstinputlisting[%
    firstline=41, lastline=44,
    language=python,
    caption={Example usage of ``numpy.einsum'' function (for Eq. (\ref{eq:jh})).},
    label={lst:einsum}%
    ]{contraction.py}
\end{center}

\subsection{Multiply composite map}
Let $T$ be a composite map of $T_i$
\begin{align}
    \begin{aligned}
        T &= T_{m-1} \circ T_{m-2} \circ \cdots \circ T_1 \circ T_0,
        &
        T: & M_0 \to M_{m}
    \end{aligned}
    \end{align}
where $T_i$ is a $C^{\infty}$ diffeomorphism $T_i: M_i \to M_{i+1}$ and $M_i \subset \R$.
Given $\bm x_i \in M_i$, the Jacobian matrix of $T$ is
\begin{align}
    \jac = \pderiv{T}{\bm x_0} = \prod_{k=0}^{m-1} \pderiv{T_{m-1-k}}{\bm x_{m-1-k}}.
\end{align}
Let us denote the product in the right-hand of equation as $\jac_{m-1}$, and we have
\begin{align}
    \begin{aligned}
        \jac_{m-1} &= \pderiv{T_{m-1}}{\bm x_{m-1}}\, \jac_{m-2}
        & \in M_{m} \otimes M_0^*,
    \end{aligned}
    \label{eq:jac}
\end{align}
for $m \ge 2$.
If $\jac_{k}$ is
Hessian tensor of $T$ is available by differentiating Eq. (\ref{eq:jac}),
\begin{align}
    \hes = \pderiv{\jac}{\bm x_0}
    =
    \parens{
        \pderiv[2]{T_{m-1}}{\bm x_{m-1}}\,
        \jac_{m-2}
    }
    \jac_{m-2}
    +
    \pderiv{T_{m-1}}{\bm x_{m-1}}
    \pderiv{\jac_{m-2}}{\bm x_0}.
\end{align}
Rewriting the derivative of $\jac_{m-1}$ by $\hes_{m-1}$, we get
\begin{align}
    \begin{aligned}
        \hes_{m-1} &=
        \parens{
            \pderiv[2]{T_{m-1}}{\bm x_{m-1}}\,
            \jac_{m-2}
        }
        \jac_{m-2}
        +
        \pderiv{T_{m-1}}{\bm x_{m-1}}
        \hes_{m-2} & \in M_m \otimes M_0^* \otimes M_0^*
    \end{aligned}
    .\label{eq:hes}
\end{align}
Notice that $\parens{\jac_{m-2}\, \jac_{m-2}}$ is sometimes not $\jac_{m-2}^2$
because the dimensions of $M_{i}$ and $M_{i+1}^*$ are not necessarily equal.

In the Python implementation with NumPy, one can write Eqs. (\ref{eq:jac}) and (\ref{eq:hes})
in oneline coding, as shown in Lst. \ref{lst:jaches_multi}.
Remember that the listing is just an example of calculation ways
but is not directly implemented in the package.
In the example, we assume the composite map composed of 8 mappings which have random dimensions.
The code implements the calculation of Eqs. (\ref{eq:jac}) and Eqs. (\ref{eq:hes})
using dummy values of Jacobian matrices and Hessian tensors.
The output example included below implies the dimension correspondence
before and after the calculation.
All Jacobian matrices transform from $M_i \otimes M_0^*$ to $M_{i+1} \otimes M_0^*$;
all Hesssian tensors do from $M_0^* \otimes M_{i} \otimes M_0^*$ to
$M_0^* \otimes M_{i+1} \otimes M_0^*$.
Notice that we swap the axes of the Hessian tensor from $M_{i} \otimes M_0^* \otimes M_0^*$
to $M_0^* \otimes M_{i} \otimes M_0^*$, to simplify the code of the recurrence relation.
\begin{center}
    \lstinputlisting[%
    firstline=21,
    language=python,
    caption={Example implementation of the recurrence relations (\ref{eq:jac}) and (\ref{eq:hes}).},
    label={lst:jaches_multi}%
    ]{composite.py}
\end{center}


\clearpage
\section{Derivative of a map for the continuous-time systems}
Consider a $C^{\infty}$ autonomous dynamical system
\begin{align}
    \begin{aligned}
        \deriv{\bm x}{t} &= \bm f(\bm x), &
        \bm x &\in M \subset \R^n, &
        t &\in \R,
    \end{aligned}
    \label{eq:sys}
\end{align}
where $M$ is a state space, and $\bm f$ is a function such that $M \to X \subset \R^n$.
We write the trajectory of the system (\ref{eq:sys}) by $\traj: X \times M \to M$,
where $\traj(0, \bm x_0) = \bm x_0$ is the initial state
and $\traj(t, \bm x_0)$ is the state at $t$.
Let $\parm$ be an $n-1$ dimensional manifold defined by
a conditional function $q: M \to \R$
\begin{align}
    \parm = \set{
        \bm x \in M ~|~
        q(\bm x) = 0
    }
\end{align}
Suppose that $T_0$ is a local map from $\bm x_0 \in M$ to a point in $\parm$
such that $M \to \parm$.
Then, the Jacobian matrix of $T_0$ is described by
\begin{align}
    \jac_{T_0} = \pderiv{T_0}{\bm x_0} =
    \left.
    \brackets{
        I -
        \frac{1}{\deriv{q}{\bm x} \bm f(\bm x)}
        \parens{\bm f(\bm x) \otimes \deriv{q}{\bm x}}
    }
    \right|_{\bm x = \bm x_1}
    \pderiv{\traj}{\bm x_0}(\tau)
    =
    B(\bm x_1) \pderiv{\traj}{\bm x_0}(\tau),
\end{align}
where $I$ is an identity matrix, $\tau$ is the spent time during the trajectory $\traj$
moves from $\bm x_0$ to the boundary $\parm$, which only depends on $\bm x_0$,
and $\bm x_1 = \traj(\tau, \bm x_0)$.
The partial derivative of the trajectory with respect to the initial state
at the time $t$ is the solution of the following ordinary differential equation
\begin{align}
    \begin{aligned}
        \deriv{}{t}
        \parens{\pderiv{\traj}{\bm x_0}}
        &= \deriv{\bm f}{\bm x} \pderiv{\traj}{\bm x_0},
        & \deriv{\traj}{\bm x_0}(0) = I.
    \end{aligned}
\end{align}

The Hessian tensor of $T_0$ is the derivative of $\jac_{T_0}$ with respect to the initial state
\begin{align}
    \begin{aligned}
        \hes_{T_0}
        &=
        \parens{\pderiv{B}{\bm x}(\bm x_1) \jac_{T_0}}\pderiv{\traj}{\bm x_0}(\tau)
        +
        B(\bm x_1)\brackets{
            \pderiv[2]{\traj}{\bm x_0}(\tau)
            +
            \parens{\deriv{}{t} \pderiv{\traj}{\bm x_0}(\tau)} \otimes \pderiv{\tau}{\bm x_0}
        },
        \\
        &=
        \parens{\pderiv{B}{\bm x}(\bm x_1) \jac_{T_0}}\pderiv{\traj}{\bm x_0}(\tau)
        +
        B(\bm x_1)\brackets{
            \pderiv[2]{\traj}{\bm x_0}(\tau)
            -\frac{1}{\deriv{q}{\bm x}\bm f(\bm x_1)}
            \parens{\deriv{\bm f}{\bm x}\pderiv{\traj}{\bm x_0}(\tau)} \otimes
            \parens{\deriv{q}{\bm x} \pderiv{\traj}{\bm x_0}(\tau)}
        },
    \end{aligned}
\end{align}
where
\begin{align}
    \pderiv{B}{\bm x}
    =
    - \frac{1}{
        \parens{\deriv{q}{x} \bm f(\bm x)}^2
    }
    \braces{
        \parens{
            \deriv{\bm f}{\bm x} \otimes \deriv{q}{\bm x}
            +
            \bm f(\bm x) \otimes \deriv[2]{q}{\bm x}
        }
        \deriv{q}{\bm x} \bm f(\bm x)
        -
        \parens{\bm f(\bm x) \otimes \deriv{q}{\bm x}} \otimes
        \parens{
            \deriv[2]{q}{\bm x} \bm f(\bm x)
            +
            \deriv{q}{\bm x} \deriv{\bm f}{\bm x}
        }
    }.
\end{align}
Notice that
\begin{align}
    \begin{aligned}
        B &\in X \otimes X^* ,&
        \pderiv{B}{\bm x} &\in X \otimes X^* \otimes X^*, &
        \deriv{q}{\bm x}&\in X^*, \;
        \deriv[2]{q}{\bm x} \in X^* \otimes X^*, &
        \bm f(\bm x) &\in X, \;
        \deriv{\bm f}{\bm x} \in X \otimes X^*,\\
        \jac_{T_0} &\in X \otimes M^*, & \pderiv{\traj}{\bm x_0} & \in X \otimes M^*, &
        \pderiv[2]{\traj}{\bm x_0} &\in X \otimes M^* \otimes M^*, &
        \hes_{T_0} &\in X \otimes M^* \otimes M^*.
    \end{aligned}
\end{align}
The second partial derivative of $\traj$ with respect to $\bm x_0$ is the solution of the following
ordinary differential equation
\begin{align}
    \begin{aligned}
        \deriv{}{t}
        \parens{\pderiv[2]{\traj}{\bm x_0}}
        &=
        \parens{\deriv[2]{\bm f}{\bm x} \pderiv{\traj}{\bm x_0}}
        \pderiv{\traj}{\bm x_0}
        +
        \deriv{\bm f}{\bm x} \pderiv[2]{\traj}{\bm x_0},
        &
        \pderiv[2]{\traj}{\bm x_0} = \bm 0,
    \end{aligned}
\end{align}
where $\bm 0$ is a tensor with all elements of zeros, and
\begin{align}
    \begin{aligned}
        \deriv[2]{\bm f}{\bm x} \in X \otimes X^* \otimes X^*.
    \end{aligned}
\end{align}

\end{document}
